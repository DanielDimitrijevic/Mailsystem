\documentclass[a4paper,12pt]{scrreprt}
\usepackage[T1]{fontenc}
\usepackage[utf8]{inputenc}
\usepackage[ngerman]{babel}
\usepackage[table]{xcolor}% http://ctan.org/pkg/xcolor
\usepackage{tabu}
\usepackage{graphicx}
\usepackage{lmodern}

\begin{document}


%\titlehead{Kopf} %Optionale Kopfzeile
\author{Dominik Backhausen \and Daniel Dimitriejvic} %Zwei Autoren
\title{  Mailsystem } %Titel/Thema
\subject{APR} %Fach
\subtitle{Aufgabe 06 } %Genaueres Thema, Optional
\date{\today} %Datum
\publishers{5AHITT} %Klasse

\maketitle
\tableofcontents


\chapter{Aufgabenstellung}
Erstelle eine Web-Applikation welche Email zugänglich macht.

    Erstelle ein Hauptmenü (/email), indem alle weiteren Punkte aufrufbar sind

    Verschicken eines Emails (/email/send):
        Eingabe der notwendigen Felder (from, to, subject, emailContent [,smtp])
        Verarbeitung der Daten, Überprüfung nach Plausibiliät und Verschicken des Emails als Servlet mittels JavaBean
        Gib eine Erfolgs- bzw. Misserfolgsmeldung aus.
        Zusatzpunkte: Attachment anfügen
    Empfangen von Emails (/email/receive):
        Eingabe der notwendigen Felder (user, password [,pop3])
        Verarbeitung der Daten, Überprüfung nach Plausibiliät und Auruf des POP3-Servers als Servlet mittels JavaBean
        Anzeigen vorhandener Emails (from, to, subject, emailContent)
        Anzeigen und Speichern der Attachments
    Die GUI's sollten einem User Acceptance Test standhalten, gut strutkturiert und in angenehmer Farbe gehalten sein!
    Die Applikation sollte dem MVC-Pattern entsprechen!
    Achte besonders auf eine ausführliche Dokumentation der Sourcen und des web deployment descriptors (web.xml).

Für die Darstellung sollte ein Facelet-Template erstellt werden.
Mithilfe dieses Templates werden die anderen Seiten erstellt. Die Verarbeitung basiert auf JSF und ManagedBeans!
    Die Applikation läuft im SessionScope
        Die Inhalte der Eingabefelder werden bei neuerlichem Formularaufbau bereits vorausgefüllt!

Abgabe:
    Gesamtes Projekt inklusive Dokumentation und *.war


Punkte (16):
Ausführliche Sourcedokumentation (4), Ressourcennutzung (4), Undeployment (2), Deployment (4), Protokoll (2 .. -4)

Diese Aufgabenstellung ist als Gruppenarbeit (2 Mitglieder) zu realisieren! Damit gelten wieder die Metaregeln in Bezug auf die Dokumentation!
\chapter{Designüberlegungen}
	
\chapter{Arbeitsaufteilung}
	\tabulinesep = 4pt
	\begin{tabu}  {|[2pt]X[2.5,c] |[1pt] X[4,c] |[1pt]X[1.3,c]|[1pt]X[c]|[2pt]}
		\tabucline[2pt]{-}
		Name & Arbeitssegment & Time Estimated & Time Spent\\\tabucline[2pt]{-}
		
		Dominik Backhausen & Testfälle & 3h & 4h\\\tabucline[1pt]{-}
		Daniel Dimitrijevic & Testfälle & 3h & 5h\\\tabucline[1pt]{-}
		Daniel Dimitrijevic & Protokoll & 1h & 1h\\\tabucline[2pt]{-}
		Gesamt && 7h & 10h\\\tabucline[2pt]{-}
	\end{tabu}	
\chapter{Arbeitsdurchführung}


\chapter{Testbericht}
\chapter{Quellen}


\end{document}